% Preamble
\documentclass[11pt]{article}

% Packages
\usepackage{amsmath}
\begin{document}

    \section{Executor Framework}\label{sec:executor-framework}

    \begin{listing}
        \item FixedThreadPool serve a creare un pool di thread con un numero fisso di thread.
        Bisogna stare attenti a non eseguire task che richiedono un tempo di esecuzione molto lungo,
        perch\`e potrebbero bloccare il pool di thread.
        \item CachedThreadPool serve a creare un pool di thread con un numero variabile di thread,
        nessun thread viene mai terminato a meno che non sia stato chiesto esplicitamente.
        \item  SingleThreadExecutor serve a creare un pool di thread con un solo thread,
        il thread viene mai terminato a meno che non sia stato chiesto esplicitamente.
        \item ScheduledThreadPool serve a creare un pool di thread con un numero fisso di thread,
        che possono essere schedulati per essere eseguiti dopo un certo periodo di tempo o periodicamente.
        A differenza di fixedThreadPool, questo pool di thread pu\`o essere usato per eseguire task periodici.
    \end{listing}

    \subsection{EXECUTOR SERVICEe}\label{subsec:callable-e-future}

    L'ExecutorService \`e un'interfaccia che estende l'interfaccia Executor,
    che fornisce metodi per la gestione di un pool di thread.
    \`E possibile terminare un ExecutorService con il metodo shutdown() o shutdownNow().
    Mi permettono di dare una sincronizzazione tra il thread principale e il thread di esecuzione tramite
    il metodo awaitTermination() che attende che tutti i task siano terminati (come se fosse un join()).

    L\'executor service \`e asincrono, perch\`e non \`e possibile sapere quando un task \`e stato eseguito.
    Per questo motivo \`e possibile utilizzare il metodo submit() che ritorna un oggetto Future.
    Il metodo get() di Future ritorna il risultato del task, se il task non \`e ancora stato eseguito,
    il metodo get() attende che il task sia eseguito.

    Quanti task mettere in campo va deciso in base al numero di core del processore durante la fase di progettazione.
    La callable e la future sono utili per avere un controllo maggiore sulle operazioni asincrone.
    La callable permette di ritornare un valore, mentre la runnable non permette di ritornare un valore.
    Quindi la callable pu\`o essere usata per eseguire operazioni che ritornano un valore, mentre la runnable pu\`o essere usata per eseguire
    operazioni che non ritornano un valore.

    \subsection{Fork - Join Framework}\label{subsec:fork-join-framework}

    I fork-join framework sono un'alternativa ai thread per la gestione delle operazioni asincrone.
    I fork-join framework sono pi\`u efficienti dei thread, perch\`e i thread hanno un costo di creazione pi\`u alto.

    \subsection{CompletableFuture}\label{subsec:completable-future}
    Le CompletableFuture sono un'alternativa ai fork-join framework.
    Le CompletableFuture sono pi\`u semplici da usare dei fork-join framework, perch\`e non \`e necessario implementare un'interfaccia.
    Estendendo la classe CompletableFuture, \`e possibile implementare un metodo che ritorna un CompletableFuture.
    Mandiamo in esecuzione il task asincrono con il metodo runAsync() o supplyAsync().

\end{document}